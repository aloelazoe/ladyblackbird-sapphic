\phantom{.}
	
\ClearShipoutPicture 
\AddToShipoutPicture{%
 	\AtPageLowerLeft{\includegraphics[width=\paperwidth,height=\paperheight,page=9]{./notext_lb.pdf}}
}

	% Header
	\placetextbox{0}{0.935}{\paperwidth}{\centering \addfontfeature{LetterSpace=0}
\textrm{\fontsize{26}{1} \ding{66}}
{\fontsize{30}{1}\scshape Running the Game}
\textrm{\fontsize{26}{1} \ding{66}}}
	\placetextbox{0}{0.9}{\paperwidth}{\fontsize{14}{1}\centering\addfontfeature{LetterSpace=0}\itshape Tips, tricks, and advice for the GM}
		\placetextbox{0}{0.465}{\paperwidth}{\centering
\textrm{\fontsize{20}{1} \ding{66}}
{\fontsize{24}{1}\scshape obstacles \& difficulties}
\textrm{\fontsize{20}{1} \ding{66}}}
	
	% Top half
	\placetextbox{0.089}{0.87}{0.4\paperwidth}{\fontsize{8}{10}\fontcelestiaantiqua
{\fontsize{12}{0}\fontcelestiaantiqua\scshape Listen \& Ask Questions, Don’t Plan} \\
When you’re the GM, don’t try to plan what will happen. Instead, ask
questions—lots and lots and make them pointed toward the things
you’re interested in. Like, Rhea gives Naomi an order within earshot
of Lady Blackbird, but the Lady’s player doesn’t register it right away.
Naomi goes to follow the order. So I ask Lady Blackbird’s player, “How
do you react when the Captain orders your bodyguard around? Is that
okay with you?” And then, when it’s totally not okay, “What do you say to
her? What do you say to Naomi?” and a few more like that and everyone
is yelling at each other and rolling dice to impose their will. \vspace{5pt} \\
Also ask questions like: \vspace{5pt} \\
\phantom{...} “Does anything break when you do this crazy maneuver?” \vspace{5pt} \\
\phantom{...} “The fire probably spreads out of control doesn’t it?” \vspace{5pt} \\
\phantom{...} “That sounds like a bold plan. What’s the first step?” \vspace{5pt} \\
\phantom{...} “Do the two of you end up somewhere quiet together? Does something \linebreak
\phantom{...} happen between you?” \vspace{5pt} \\
\phantom{...} “Do you know anything about the Crimson Sky rebels? What are \linebreak
\phantom{...} they like? Is it normal for them to be this far into the Empire?” \vspace{5pt} \\
Keep that going at a steady pace and the game flies along pretty well.
Part of the job of the GM is listening to what the players say, catching it,
turning it around and looking at it, and seeing if there’s anything else to
be done with it. \vspace{5pt} \\
\textbf{The GM’s jobs}: listen and reincorporate, play the NPCs with gusto,
create interesting obstacles, and impose conditions as events
warrant \\
\phantom{.....} (especially when rolls fail).
}


\placetextbox{0.51}{0.87}{0.4\paperwidth}{\fontsize{8}{10}\fontcelestiaantiqua
{\fontsize{12}{0}\fontcelestiaantiqua\scshape Say Yes, Look For the Obstacles} \\
By default, characters can accomplish anything covered by their traits.
They’re competent and effective people, in other words. It’s no fun to
ask for a roll when there’s no cool obstacle in the way. Just say yes to the
action, listen, and ask questions as usual. But also, be on the look out for
the opportunity to create obstacles as the action develops. Because you’re
asking leading questions and listening closely, they’ll be all over the place,
so it won’t be too hard to spot them. \vspace{5pt} \\
Obstacles can be people (pirates, goblins, imperials, citizens, nobles),
weather, monsters (sky squid, flying eels), situations (fires, falling, being
shot at, chases, escapes) or anything else you can imagine. \vspace{5pt} \\
If a character tries something not covered by their traits, that’s an
obstacle right there: lack of experience and training. Lots of fun things
can go wrong when you don’t know what you’re doing! Also, players will
sometimes try things they’re bad at so they can fail and add dice to their
pool. It’s a fine move for them and it gives you the chance to create more
trouble, so everyone wins. \vspace{10pt} \\
{\fontsize{12}{0}\fontcelestiaantiqua\scshape Conditions} \\
A condition constrains what the player should say about their character.
It’s a cue to tell the GM and players to pay attention to that thing and
use it as material for the developing fiction. Gaming is just us saying stuff
to each other, right? So you’re like, “What do I say now?” and you look
down and go, “Oh, I’m angry. I’ll go be angry at someone then. ‘Bethelblab!
Why aren’t we at Nightport yet, you shiftless layabout?!’” For the GM,
the conditions can create opportunities or give permissions. “You’re
Injured, right? The Void Spiders can smell blood. They swarm right
at you, ignoring the others.” Sometimes a condition will  \\
become an Obstacle in its own right, calling for a roll to deal \\
with it.
}
	
	
	% Bottom half
	
	\placetextbox{0.09}{0.435}{0.3\paperwidth}{\fontsize{12}{0}\fontcelestiaantiqua Escape the Brig}
	\placetextbox{0.09}{0.24}{0.3\paperwidth}{\fontsize{12}{0}\fontcelestiaantiqua Bounty Hunter Ambush}
	\placetextbox{0.37}{0.435}{0.3\paperwidth}{\fontsize{12}{0}\fontcelestiaantiqua Skyship Battle}
	\placetextbox{0.37}{0.235}{0.3\paperwidth}{\fontsize{12}{0}\fontcelestiaantiqua Parlay with Scoundrels}
	\placetextbox{0.65}{0.435}{0.3\paperwidth}{\fontsize{12}{0}\fontcelestiaantiqua Sky Squid Attack}
	\placetextbox{0.65}{0.255}{0.3\paperwidth}{\fontsize{12}{0}\fontcelestiaantiqua Fight a Sorceress}
	
	
	\placetextbox{0.10}{0.415}{0.26\paperwidth}{\fontsize{9}{0}\fontcelestiaantiqua 
\textit{The cells in the Hand of Sorrow brig are walled
in steel with heavy iron locks on the doors.} \vspace{5pt} \\
\textsc{Obstacles}: Pick the lock: 3. Trick a marine
guard: 3. (Bishop only—Smash the door
open: Automatic. Smash the door open
quietly: 5) Sneak through the ship: 4. Fight
crew: 3. Fight marines: 4. Fight a lot of
marines: 5 (or higher). \vspace{5pt} \\
\textsc{Escalation}: Alarm goes off. More marines
appear. The Owl is jettisoned to stop your
escape. Someone gets separated from the
group (Lost and/or Trapped).
	}
	
	\placetextbox{0.10}{0.22}{0.26\paperwidth}{\fontsize{9}{0}\fontcelestiaantiqua 
\textit{Unless they keep a low profile, the actions of The
Owl will eventually draw attention from bounty
hunters looking to cash in on the warrants for
Vance or the reward for Lady Blackbird.} \vspace{5pt} \\
\textsc{Obstacles}: Fight back when ambushed: 5.
Flee: 3. Try to bargain with them: 4. Pull a
dirty trick to turn the tables: 3. \vspace{5pt} \\
\textsc{Escalation}: Someone is grabbed and held
at gunpoint (Trapped).
	}
	
	\placetextbox{0.38}{0.415}{0.26\paperwidth}{\fontsize{9}{0}\fontcelestiaantiqua 
\textit{You always want to be above your enemy in a
skyship battle—unless your vessel is equipped to
go into the Lower Depths....} \vspace{5pt} \\
\textsc{Obstacles}: Maneuver for a clear shot: 3.
Maneuver against a smaller, faster ship: 4.
Maneuver to boarding action: 4. Fire on enemy
ship: 3. Fire on a smaller, faster ship: 4. Avoid
enemy fire: 3. Avoid a lot of enemy fire: 4-5. \vspace{5pt} \\
\textsc{Escalation}: \textit{The Owl} is hit and loses control
(Busted \& Leaking, Slowed). More enemy
ships appear. You’re driven into a storm by
enemy action. The fight attracts a sky squid.
	}
	
	\placetextbox{0.38}{0.215}{0.26\paperwidth}{\fontsize{9}{0}\fontcelestiaantiqua 
\textit{To find the secret path to the lair of the Pirate
queen in the remnants, you’ll have to deal with a
whole host of unsavory characters.} \vspace{5pt} \\
\textsc{Obstacles}: Find an underworld den: 3.
Show you’re not someone to mess with: 3.
Arrange a fair deal: 4. Arrange a deal that goes
in your favor: 5. Spot their devious lies: 4. \vspace{5pt} \\
\textsc{Escalation}: The scoundrels decide to
simply take what they want from you. You’re
sold out. You were followed to the meet!
	}
	
	\placetextbox{0.66}{0.415}{0.26\paperwidth}{\fontsize{9}{0}\fontcelestiaantiqua 
\textit{While passing through the lower depths, your
engines attract a hungry sky squid. Its tentacles
close around The Owl....} \vspace{5pt} \\
\textsc{Obstacles}: Escape from tentacles: 5.
Attack Squid: 3. Maneuver in squid ink: 4.
Avoid harm from squid attacks (crushing,
smashing, biting, thunderous song): 3. \vspace{5pt} \\
\textsc{Escalation}: Squid calls other squid
with its song. Squid blood attracts other
monster(s). Pulled further into the depths
(Lost). Crash into rocks/debris/hidden
world (Busted \& Leaking or Crippled).
	}
	
	\placetextbox{0.66}{0.235}{0.26\paperwidth}{\fontsize{9}{0}\fontcelestiaantiqua 
\textit{Aria Flame is a flameblood and a powerful
sorceress. Not that anyone would need to fight
her, though. I mean, why would that happen?} \vspace{5pt} \\
\textsc{Obstacles}: Dodge blasts of magical fire: 3.
Attack Flame through her magical defenses: 5.
Endure the heat and smoke as the fight
wears on: 3. \vspace{5pt} \\
\textsc{Escalation}: The fires spread out of control.
You drop your weapons when they get too
hot to hold.
	}
	
	
	
	
	
	
	
	
	
	
	
	
	
	
	